\beginsong{Šel nádražák na mlíčí}[by={Jára Cimrman}]
\beginverse
V šestnáct nula pět, když projel nákladní vlak na Prahu, uchopil hradlář Karel Berka nůž, opustil pracoviště a vydal se po trati. Byl klidný, slunečný den. Traťový svršek se zelenal bohatými trsy odkvetlých pampelišek. 
\endverse
\beginref
\[C]V Poříčí, v Poříčí, šel nádražák \[G]na mlíčí
mezi pražce, mezi praž\[G]ce.
\[C]Než příští vlak profičí, bude píce \[G]králičí
dávno v tašce, dávno v taš\[C]ce.
\lrep \[G]Na železnici, \[C]dějou se věci,
\[G]na dráze jsou zaměstnáni \[C]švarní mládenci. \rrep
\endverse
\beginverse
Jak Berka podřezával trsy mlíčí, všiml si, že není na trati sám. Nedaleko vjezdového návěstidla spatřil Marii Rezkovou, která pevně rozkročena v kolejišti plnila mlíčím svůj proutěný koš. 
\endverse
\beginref
Nádražák zakřičí: Kam to chodíš na mlíčí? 
Kam to chodíš na lupení? 
Starej se o sebe, řekla Madla pro sebe, 
ale Berka hluchý není.
\lrep Na železnici\ldots \rrep
\endverse
\beginverse
Berka Rezkové po dobrém vysvětlil, že na mlíčí, které je u dráhy, má nárok on, protože je také u dráhy, ale žena se dál oháněla svou kudličkou. Když mu doslova pod rukou odřízla nejlepší trs, Berka zrudl a v slunci se zablýskl jeho nůž.
\endverse
\beginref
Ubožáku zarudlý, chceš-li, tak pojď na kudly, 
ajznboňácká ty zmije. 
Dráze patřej mašiny, všecky pražce a šíny, 
ale mlíčí obecní je. 
\lrep Na železnici\ldots \rrep
\endverse
\beginverse
Tou dobou jel zmíněnou tratí ve své služební drezíně inspektor Jihozápadní dráhy Kulihrach. Byl to čerstvý vdovec a kochal se pohledem na tu naši malebnou českou krajinu. Ještě že si vybral inspekční cestu právě dnes.
\endverse
\beginref
Na nože, na nože, nedošlo jen proto, že 
zčistajasna vjel mezi ně 
inspektor státních drah, doktor Václav Kulihrach, 
na drezíně, na drezíně.
\lrep Na železnici\ldots \rrep
\endverse
\beginverse
Kdo z nás Čechů není chovatel? Ani inspektor nebyl výjimkou, a proto měl pro oba zápasící pochopení. Aby zabránil dalším sporům, všechno mlíčí zabavil, neboť měl doma malochov statných českých strakáčů a své inspekční cesty rád spojoval s cestou za krmením. 
\endverse
\beginref
Povídá: Ty, Kadle, ty máš službu na hradle, 
ať už jsi tam, ať už jsi tam. 
Vás, Madlo, vás, Madlo, s očima jak zrcadlo, 
doufám víckrát nenachytám.
\lrep Na železnici\ldots \rrep
\endverse
\beginverse
Nad pešuňkem se rozhostil vlahý podvečer. Kolejnice vyhřáté odpoledním sluncem se leskly a vzduch nad nimi se tetelil. Telegrafní sloupy podél trati hučely svou monotónní píseň dálek, doprovázenou staccatem cvrčků a zvonkohrou chráněného železničního přejezdu. 
\endverse
\beginref
Dobře to dopadlo. Berka spěchá na hradlo, 
aby pohnul semaforem. 
Když návěst přepíná, jede kolem drezína 
a v ní Madla s inspektorem. 
\lrep Na železnici\ldots \rrep
\endverse
\endsong