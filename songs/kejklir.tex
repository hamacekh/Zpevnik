\beginsong{Kejklíř}[by={Luboš Odháněl}]
\beginverse
\[Ami]Kdysi jsem na zámku hrál na svou kytaru,
\[C]kdysi jsem \[Dmi]kejklířem \[Ami]byl.
\[Ami]Bavil jsem krále i jeho smetánku,
\[F]pokorný \[Emi]život jsem \[Ami]žil.
Než jsem se \[G]zahleděl do očí \[Ami]tvých,
do tváře \[Dmi]bílé \[C]jak padlý \[G]sníh.
Před branou \[Ami]štěstí, \[F]hm, \[Dmi]chvíli jsem \[G]stál,
\[F]když jsem se za\[(Emi)]milo\[Ami]val. \[Dmi]Há-ha-ha-ha \[G]Há
\[F]Když jsem se za\[(Emi)]milo\[Ami]val.. 
\endverse
\beginverse 
Ten král se však rozhněval, že dívka mě ráda má,
že lásku svou dokáže dát.
Pryč táhni mi ze zámku, kejklíři špinavá,
a k vojsku mě zverbovat dal.
A já teď slyším Tvůj nářek a pláč
a Ty se, lásko, zoufale ptáš:
Proč odcházím do boje, hm, bodákům vstříc
a nemůžu na to nic říct. Há-ha-ha-ha Há
a nemůžu na to nic říct.
\endverse
\beginverse
Pět let jsem v legii za právo bojoval,
pět let jsem o Tobě snil.
A smrtce v bílém hávu jsem vzdoroval,
a přesto mi zbylo dost sil.
Pozvednout kytaru a s ní i hlas,
že trocha lásky přec zbylo v nás.
Tak Ti ji posílám, hm, na strunách zní,
snad k Tobě včas doletí. Há-ha-ha-ha Há
snad k Tobě včas doletí.
\endverse
\beginverse
Já dočkal se vítězství a k Tobě domů hnal,
zklamání jsem se dočkal.
Na nádvoří v obětí drží Tě sám pan král
a děcko hned jsem poznal.
Proč osud dokáže tak krutý být
a z kalichu smutku já zas musím pít.
Proč oni jsou nahoře, hm, a já zas níž,
lásko, co o tom víš? Há-ha-ha-ha Há
lásko, co o tom víš?
\endverse
\beginverse
Tak zvednul jsem hlavu výš, nu, což mi zbývalo,
a zabalil kytaru svou.
Králi se uklonil, sehnuv’ se trochu níž,
jsem kejklíř a život je hrou.
A zas je mým domovem jen širá pláň,
lásce i osudu zaplatím daň.
Před branou štěstí, hm, chvíli jsem stál,
než jsem se zamiloval. Há-ha-ha-ha Há
než jsem se zamiloval.
\endverse
\endsong
