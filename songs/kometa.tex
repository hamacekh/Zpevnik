\beginsong{Kometa}[by={Jaromír Nohavica}]
\beginverse
S\[Ami]patřil jsem kometu, oblohou letěla,
chtěl jsem jí zazpívat, ona mi zmizela,
z\[Dmi]mizela jako laň u \[G7]lesa v remízku,
v \[C]očích mi zbylo jen \[E7]pár žlutých penízků.
\endverse
\beginverse
Penízky ukryl jsem do hlíny pod dubem,
až příště přiletí, my už tu nebudem,
my už tu nebudem, ach, pýcho marnivá,
spatřil jsem kometu, chtěl jsem jí zazpívat.
\endverse
\beginref
\[Ami]O~vodě, o~trávě, \[Dmi]o~lese,
\[G7]o~smrti, se kterou smířit n\[C]ejde se,\[E]
\[Ami]o~lásce, o zradě, \[Dmi]o~světě
\[E]a o všech lidech, co \[E7]kdy žili na téhle pl\[Ami]anetě.\[Emi]
\endverse
\beginverse
Na hvězdném nádraží cinkají vagóny,
pan Kepler rozepsal nebeské zákony,
hledal, až nalezl v hvězdářských triedrech
tajemství, která teď neseme na bedrech.
\endverse
\beginverse
Velká a odvěká tajemství přírody,
že jenom z člověka člověk se narodí,
že kořen s větvemi ve strom se spojuje
a krev našich nadějí vesmírem putuje.
\endverse
\beginref
Na na na\ldots
\endverse
\beginverse
Spatřil jsem kometu, byla jak reliéf
zpod rukou umělce, který už nežije,
šplhal jsem do nebe, chtěl jsem ji osahat,
marnost mne vysvlékla celého donaha.
\endverse
\beginverse
Jak socha Davida z bílého mramoru
stál jsem a hleděl jsem, hleděl jsem nahoru,
až příště přiletí, ach, pýcho marnivá,
my už tu nebudem, ale jiný jí zazpívá.
\endverse
\beginref
O vodě, o trávě, o lese,
o smrti, se kterou smířit nejde se,
o lásce, o zradě, o světě,
bude to písnička o nás a kometě\ldots
\endverse
\endsong
