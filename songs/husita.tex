\beginsong{Husita}[by={Jaromír Nohavica}]
\beginverse
\lrep \[C]Pásával jsem koně \[G]u~nás ve dvoře, 
\[C]ale už je nepa\[G]su,\rrep
\[Ami]chudák, ten je \[G]dole, \[C]a~pán naho\[F]ře, 
\[C]všeho jenom \[G]do ča\[Ami]su, \[F]jó, vš\[C]eho jenom \[G]do ča\[C]su. 
\endverse
\beginverse
\lrep Máma ušila mi režnou kytlici, 
padla mi jak ulitá,\rrep
táta vytáh' ze stodoly sudlici: 
\uv{teď jseš, chlapče, husita, jó, teď jseš, chlapče, husita.}
\endverse
\beginref
\lrep \[C]Hejtman volá:\uv{\[G]Do zbra\[C]ně, \[Ami]bijte \[Dmi]pány, \[G7]hr na \[C]ně!} \[G7 (--)]\rrep 
\[Ami]a mně \[Dmi]srdce \[G7]buší, \[C]lásce dal jsem \[Ami]duši, \[Dmi]jen ať s námi \[G7]zůstan\[C]e. 
\endverse
\beginverse
\lrep U města Tachova stojí křižáci, 
leskne se jim brnění,\rrep
sudlice je těžká, já se potácím, 
dvakrát dobře mi není, jó, dvakrát dobře mi není. 
\endverse
\beginverse
\lrep Tolik krásnejch holek chodí po světě, 
já žádnou neměl pro sebe,\rrep
tak si říkám: chlapče, křižák bodne tě, 
čistej půjdeš do nebe, jó, čistej půjdeš do nebe.
\endverse
\beginref
Hejtman volá\ldots
\endverse
\beginverse
\lrep Na vozové hradbě stojí Marie, 
mává na mě zdaleka,\rrep
křižáci, kdo na ni sáhne, mordyje, 
ten se pomsty dočeká, jó, ten se pomsty dočeká. 
\endverse
\beginverse
\lrep Chtěl jsem jí dát pusu tam, co je ten keř, 
řekla:\uv{To se nedělá,}\rrep 
když mě nezabijou, to mi, holka, věř, 
budeš moje docela, jó, budeš moje docela. 
\endverse
\beginref
Hejtman volá\ldots
\endverse
\beginverse
\lrep Už se na nás ženou křižáci smělí, 
zlaté kříže na krku,\rrep
jen co uslyšeli, jak jsme zapěli, 
zpět se ženou v úprku, jó, zpět se ženou v úprku. 
\endverse
\beginverse
V trávě leží klobouk, čípak mohl být, 
prý kardinála z Anglie, 
v trávě leží klobouk, čípak mohl být, 
prý kardinála z Itálie, 
tam v tý trávě zítra budeme se mít 
já a moje Marie, jó, 
\[C]ať miluje, \[G]kdo ži\[Ami]je, \[F]jó, 
\[C]ať žije his\[G]tori\[C]e! 
\endverse
\endsong
