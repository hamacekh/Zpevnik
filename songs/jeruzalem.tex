\beginsong{Jeruzalém}[by={Jaromír Nohavica}]
\beginref
\memorize
\[C]Koupil jsem \[C7]bílý papír,
\[F]napsal naň, \[Dmi]to co mě \[G]trápí,
\[C]najal jsem loď a vy\[G]plul do Jeruza\[C]léma 
\endverse
\beginstar
\replay
^U zdi nářků ^papír jsem složil
^a do malé škvíry mezi ^kameny ho ^vložil
^s důvěrou v osud, který ^krouží nad náma ^všema.
\endverse
\beginverse
\[C]Na hlavu dal jsem čapku omotanou motou\[Ami]zem
a mlčky stál jsem a myslel na svou malou \[Dmi]zem,
\[]která tam na severu mezi horama skryta,
\[G]zleva i zprava bita, \[G7]shora i zdola bita.
\endverse
\beginverse
Tak jsem tam stál a myslel na ně,
jak je to těžké, stát na správné straně,
když vichry vanou nahoru i dolů,
Moravskou branou k severnímu pólu.
\endverse
\beginref
~
\endverse
\beginstar
Ruce z kapes a vlasy za límec,
sto metrů za vesnicí jsme každý cizinec,
jak dudák ze Strakonic, všechno anebo nic.
\endverse
\beginstar
\[C]O polívku řekneš si v řeči jidiš,
\[Ami]o chleba taky i když trochu se stydíš,
ale \[Dmi]písničku v cizí řeči \[G]tu si nepoří\[C]díš.
\endverse
\beginverse
A tak uprostřed světa v té cizí zemi
mraky mě obepluly melodiemi,
slyšel jsem ticho chorálů i hlas naděje,
přilétlé ze severu na křídlech tereje.
\endverse
\beginverse
Možná to byl jenom prach ta slza z mého oka,
v mysl mi vytanula písnička útloboká,
ve školním autobuse dokola jedna sloka,
holka modrooká nesedávej u potoka.
\endverse
\beginref
~
\endverse
\endsong